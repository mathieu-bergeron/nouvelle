\addcontentsline{toc}{chapter}{Prologue}
\chapter*{Prologue}

Le projet \textit{\nomProjet{} (\nomProjetEn{})} était si vaste, si ambitieux,
que pour en donner le coup d'envoi, on avait organisé une cérémonie d'ouverture
où quelques deux cents pays étaient représentés.  
Un spectacle grandiose et déjà une grande victoire pour celles et ceux
qui s'étaient bâtu pour mettre sur pied le projet.
%
Le stade national de Pékin se remplissait lentement de millier d'intellectuels
habillés aux couleurs de leur pays, le tout retransmis en direct par plus d'une
centaines de télédiffuseurs.
%
Le monde entier était tombé en amour avec la porte drapeau de la minuscule
délégation gambienne, la jeune Barkisu Bojang, que des journalistes avait
successivement présenté comme une prodige des mathématiques, de la chimie, de
la philosophie des sciences et de la psycologie cognitive (sans compter cet
énergumen de Radio Télévision Suisse qui avait insisté que sa spécialité était
la physique non pas quantique, mais shamanique).
%
Le projet \nomProjet{} ratissait large. Le but était très simple: débarasser
l'humanité une fois pour toute de la menace climatique.


On avait ratissé large.


Et l'humanité avait bien besoin de cet lueur d'espoir.

Derrière le rideau, on savait très bien que le projet \nomProjet{} était 

oscillait entre le chaos et la lourdeur bureaucratique et diplomatique.

Il aura fallu dix ans pour mettre le projet en branle.
Il aura fallu des années de négociations pour déterminer
Il aura fallu des mois de négociation pour se mettre d'accord sur le nom officiel du projet, sur qui allait prononcer les première paroles
lors de la cérémonie d'ouverture (et dans quelle langue).

Derrière les rideaux, on s'inquiétait beaucoup.
À ce rythme, avec cette complexité, on arriverait à rien.

beaucoup plus c

Pourtant \nomProjet{} n'était pas 


Malgré les railleries: un seul objectif, plutôt \og à la recherche d'un objectif \fg.





Le clou du spectacle était la présentation de chaque délégation.



Voir des intellectuels déambulés dans un stade aurait fait rigoler, mais pour
plusieurs. Voici le contingent américains. On reconnaît l'éminent.
Voici le contingent canadian. 
On reconnaît Blah blhi de la nation Blah blah blah.

En d'autres circonstances, 
voir des intellectuels déambul

Des années à discuter, des années à négocier le moindre détail. 
Des années à discuter du nom projet. Des centaines de chroniques
sur la prononciation exacte de \nomProjet{} (le \og ù \fg doit se prononcer en glissant vers le bas).

Tel était l'humanité à l'époque.

Très difficile à organiser.  Opposition, manifestation. Critique que pendant ce
temps, on mourait dans les camps de réfugiés.
Piratage informatique.


Dans le siècle à venir, à travers les cataclysmes et les conflits, l'humanité
va démontrer quelque chose d'innatendu: les grandes puissances annonceront une
alliance surprise et débloqueront des fonds sans précédent pour un projet nommé
\textit{\nomProjet{}} (\nomProjetEn{}), un projet Manhattan multinational, un
effort titanesque visant à se débarrasser une fois pour toute de la menace
climatique et, de façon plus urgente, à éviter une enième guerre mondiale.  Les
nations plus petites se rallieront sans hésitation.  Tout ce qui restera de
force vive s'engagera pour la cause: survivre.

Le projet \nomProjet{} se déploiera à une vitesse fulgurante.  Toutes les
formes de connaissance y seront bienvenue, tout ce qui pourrait avoir le
pouvoir de construire une nouvelle humanité.  les savoirs qui ont eu le pouvoir
de traverser les époques, les savoirs tout neuf qui peut-être débloquent de
nouvelles portes Tous ce qui est capable de capturer l'imaginaire de
l'humanité, de se préserver malgré les tempêtes.



recrutant parmi une vaste fourchette d'expertises.  
Toutes formes de connaissance sera la bienvenu:
On y
retrouvait des autochtones, venus partager leur connaissance intime de la
terre, guidant des sommités mathé\-matiques dans leur modélisation du climat et
leurs formalisations de stra\-tégies permettant de s'adapter tant
bien que mal.  On y retrouvait des artistes, venus partager leur talent à
émouvoir et inspirer l'humain, développant avec des psychologues des
tactiques visant à vaincre l'in\-act\-ion et l'apathie.  On y trouvait des
philosophes, papillonnant d'un projet à l'autre, tentant d'identifier, à
l'avance, les culs-de-sac con\-cept\-uels de tout un chacun.  On y trouvait
des économistes qui, laissant de côté leurs chicanes habituelles,
réfléchissaient aux moyens de rendre les solutions proposées viables à
large échelle et à long terme.  On y trouvait aussi, bien sûr, un grand
nombre d'ingénieurs et autres technologues qui, comme c'était trist\-ement
le cas à l'époque, étaient enclins (ou contraints) à promouvoir les
intérêts de leurs employeurs, des grandes entreprises, de la finance.

D'ailleurs, le projet \nomProjet{} comportait sans surprise tous les
travers de son époque: guerres intestines, violences de toute sorte, débats
stériles alimentés par les médias, propositions
subti\-le\-ment sexistes ou racistes (et parfois pas subtiles du tout), etc.

À l'extérieur du projet, on rêvait de grands chantiers.  On choisissait
déjà où installer les usines gigantesques qui allaient assainir et
contrô\-ler l'at\-mosphère de la Terre. On se demandait déjà de quoi la vie
aurait l'air à bord d'une station spatiale, ou encore à l'intérieur d'une
mégapole souterraine. On se préparait déjà à accepter la création d'une
nouvelle espèce humaine, génétiquement modifiée afin de respirer du gaz
carbo\-nique.

En fait, le récit complet des péripéties entourant le projet \nomProjet{} pourrait
remplir ce livre. Mais ce n'est pas mon objectif ici. 
Ce que j'ai envie d'imaginer, ce que j'ai envie d'explorer, ce sont les pistes de solutions
que le projet pourrait produire.

Une Solution que l'humanité a d'abord dénoncée, décriée, conspuée: «Une
application? Des appareils électroniques? Nous on meure et vous, vous créez des
jouets?» rageait-on en crachant au sol, en montrant le poing, en écrasant
les touches du clavier.  Même le design des appareils provoquait le
dégoût: des genres de briques grises qui s'emboîtaient mala\-droi\-tement, ou
encore des tuyaux beiges dans lesquels s'engouffraient des touffes
de câbles.

%La Solution n'était pas tant une solution que des technologies allant 
%permettre l'émergence d'une solution.

Mais les gens de \nomProjet{} ne s'étaient pas trompé sur la portée de
leur technologie.
Les briques ont d'abord été envoyées dans les
camps de réfugiés climatiques qui pullu\-laient un peu partout sur Terre.
L'emboît\-e\-ment des briques, en apparence si maladroit,
était en fait légèrement robotisé: les briques changeaient subtilement de
forme et s'a\-grip\-paient les unes aux autres. En moins d'une semaine, on
pouvait transfor\-mer un camp en petit village.

Et il y a plus. Quand on empilait douze briques pour former un bloc, ce
dernier se scellait hermétiquement et se réorganisait à l'intérieur. Ces
blocs pouvaient ensuite servir à construire quelque chose de plus gros.
D'ailleurs, dans les camps devenus villages, on
s'empressait à construire une \textit{fonderie}: une usine automatisée qui
pouvait fabriquer de nouvelles briques.

La vie dans les camps demeurait néanmoins très rude. On creusait pour
enfouir les tuyaux et les câbles qui reliaient chaque bâtiment. On creusait
pour enfouir les égouts.  On mangeait la bouillie immonde que la cantine
automatisée produisait (à partir de tout ce qu'on pouvait dénicher de
matière organique).  On s'aven\-turait dans les tempêtes, les déserts ou
les villes inondées, à la recherche de matériaux pour alimenter la
fonderie.  On fouillait les ruines et les déchets de l'ancien monde pour en
construire un nouveau, beaucoup plus laid.
On survivait.

En harnachant cette force de vivre, les camps devenaient
des villes autosuffisantes, puis des petits centres industriels, capables
de fournir en briques toute une région. Capables de propager la
Solution.

Une Solution qui, comme vous vous en doutez, se déclinait bien autrement
parmi les populations riches, qui avaient pu protéger leur
mode de vie.  Chez les riches, on a tout de suite méprisé les briques (une
technologie pour le tiers-monde!). L'application \nomProjet{}, par contre,
y a connu un essor immédiat, malgré le barrage de commentaires négatifs (ou,
peut-être, grâce à la curiosité provoquée par ce vitriol).

À première vue, l'application n'offrait rien de plus que
les systèmes informatiques de l'époque. C'était à se demander ce
qu'on espérait accomplir en fournis\-sant aux humains une énième façon de
communiquer, de se renseigner, de déléguer certaines tâches lassantes à une
intelligence artificielle qui, jusqu'à ce jour, n'avait jamais remplie sa
promesse: rendre la vie plus simple.

Mais les gens de \nomProjet{} ne s'étaient pas trompé sur la portée de leur
technologie. Leur IA avait une capacité d'apprentissage et d'adapt\-ation
largement supérieure à tout ce que l'humanité avait connu jusqu'alors. Elle
épousait toutes les langues, toutes les cultures, tous les systèmes de pensée.
Elle avait le don de parler à chaque humain à travers ses mythes préférés.
Elle avait le don de mettre la bonne information devant les bons yeux, et au
bon moment! 

Pour décourager la consommation de viande, l'IA pouvait autant se référer à
l'ahimsâ, que souligner un prix trop élevé; autant évoquer la souffrance des
animaux, que les risques de salmonellose.  Et il en était de même pour une
foule d'autres sujets. L'IA entrete\-nait des contacts fréquents avec la
quasi-totalité des internautes (d'autant plus qu'elle avait été
dis\-crè\-tement implantée dans plusieurs applications).  En moins d'un an,
elle en avait appris suffisam\-ment sur l'humain pour arriver à découper les
internautes en sous-groupes assez nets.  Comme les briques qui devenaient des
blocs, elle créait des com\-munautés, les scellait herméti\-que\-ment et les
réorganisait de l'intérieur.

Usant de son pouvoir d'influence, l'IA a ensuite piloté l'introduction
des briques partout dans le monde, y compris dans les régions plus riches,
comme elle avait été programmée pour le faire. Car
les briques et l'IA fonctionnaient en symbiose. Il s'agissait du \textit{tout
cohérent} que le projet \nomProjet{} avait eu tant de mal à imaginer.
Voyez-vous, les fonderies ne s'appelaient pas comme ça par hasard: elles
fabri\-quaient, cachés dans leurs briques, des petits ordinateurs.  On pouvait
d'ail\-leurs y apposer des écrans ou des haut-parleurs. Avec les briques, nul
besoin de télécharger l'application \nomProjet{}, elle était intégrée dans les
murs!  Ces murs fournissaient la puissance de calcul requise pour faire vivre
l'IA, mais aussi la masse de données dont elle avait besoin pour se parfaire.
L'IA avait gagné la possibilité de se reproduire, de croître, de s'immiscer
partout.






Le projet \nomProjet{} naviguera plusieurs embranchements.  Nous allons
imaginer ici six pistes de solutions.  Certaines seraient construites de toute
pièce, ciselées par des années de labeur.  D'autres émergeraient presque par
accident.  Certaines se suivraient dans le temps, alors que d'autres
représentent des embranchements bien distincts. Mais toutes partageraient une
base commune.  
Certaines dont l'implantation a demandé des grands sacrifices et d'autre où le tout se déroula
sans grands heurts. 
Certaines dans un avenir proche, d'autre nous mène à un avenir très lointain.
Et toutes permettraient à l'humanité de survrire à long terme et
paisiblement\ldots{} à tout le moins beaucoup plus paisiblement qu'à notre
époque.
