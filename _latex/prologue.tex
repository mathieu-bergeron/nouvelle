\addcontentsline{toc}{chapter}{Prologue}
\chapter*{Prologue}

Dans le siècle à venir, à travers les cataclysmes et les conflits, l'humanité
va démontrer quelque chose d'innatendu: les grandes puissances annonceront une
alliance surprise et débloqueront des fonds sans précédent pour un projet nommé
\textit{\nomProjet{}} (\nomProjetEn{}), un projet Manhattan multinational, un
effort titanesque visant à se débarrasser une fois pour toute de la menace
climatique et, de façon plus urgente, à éviter une enième guerre mondiale.  Les
nations plus petites se rallieront sans hésitation.  Tout ce qui restera de
force vive s'engagera pour la cause: survivre.

Le projet \nomProjet{} se déploiera à une vitesse fulgurante.  Toutes les
formes de connaissance y seront bienvenue, tout ce qui pourrait avoir le
pouvoir de construire une nouvelle humanité.  les savoirs qui ont eu le pouvoir
de traverser les époques, les savoirs tout neuf qui peut-être débloquent de
nouvelles portes Tous ce qui est capable de capturer l'imaginaire de
l'humanité, de se préserver malgré les tempêtes.



recrutant parmi une vaste fourchette d'expertises.  
Toutes formes de connaissance sera la bienvenu:
On y
retrouvait des autochtones, venus partager leur connaissance intime de la
terre, guidant des sommités mathé\-matiques dans leur modélisation du climat et
leurs formalisations de stra\-tégies permettant de s'adapter tant
bien que mal.  On y retrouvait des artistes, venus partager leur talent à
émouvoir et inspirer l'humain, développant avec des psychologues des
tactiques visant à vaincre l'in\-act\-ion et l'apathie.  On y trouvait des
philosophes, papillonnant d'un projet à l'autre, tentant d'identifier, à
l'avance, les culs-de-sac con\-cept\-uels de tout un chacun.  On y trouvait
des économistes qui, laissant de côté leurs chicanes habituelles,
réfléchissaient aux moyens de rendre les solutions proposées viables à
large échelle et à long terme.  On y trouvait aussi, bien sûr, un grand
nombre d'ingénieurs et autres technologues qui, comme c'était trist\-ement
le cas à l'époque, étaient enclins (ou contraints) à promouvoir les
intérêts de leurs employeurs, des grandes entreprises, de la finance.

D'ailleurs, le projet \nomProjet{} comportait sans surprise tous les
travers de son époque: guerres intestines, violences de toute sorte, débats
stériles alimentés par les médias, propositions
subti\-le\-ment sexistes ou racistes (et parfois pas subtiles du tout), et
j'en passes.

À l'extérieur du projet, on rêvait de grands chantiers.  On choisissait
déjà où installer les usines gigantesques qui allaient assainir et
contrô\-ler l'at\-mosphère de la Terre. On se demandait déjà de quoi la vie
aurait l'air à bord d'une station spatiale, ou encore à l'intérieur d'une
mégapole souterraine. On se préparait déjà à accepter la création d'une
nouvelle espèce humaine, génétiquement modifiée afin de respirer du gaz
carbo\-nique.

Au final, le projet \nomProjet{} produira six pistes de solution.
Certaines construites de toute pièce, ciselé par des années de labeur.
D'autres qui émergeront presque par accident.
Ce livre raconte chacune de ces six pistes.
