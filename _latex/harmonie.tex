\chapter{Harmonie}

Sur la place du village trônait une grande fresque mécanique, actionnée par
plusieurs petits moulins à vent.
%
Selon la force et la direction du vent, les 23 personnages de la fresque
(Étienne les avait compté) changaient de place, se déplaçaient sur des rails
différents, aggripaient des outils différents.  
%
À chaque jour on pouvait voir une autre histoire... et entendre une nouvelle
mélodie!  
%
Une douce musique, un cliquetis de bois sur bois où chaque pièce
avait été calibré à la main et s'harmonisait.

Étienne avait fait rire sa grand-mère le jour où il avait demandé à quoi la
fresque pouvait bien servir.

-- À quoi ça sert? Tu es bien comme moi Étienne. La fresque n'a pas besoin de
servir à quelque chose. Elle est là pour émerveiller. Elle est là pour être
belle. Savais-tu que cette fresque est à peine plus vieille que toi. On l'a
bâtie en arrivant ici, après que la grande tempête eut emporté 
notre dernier village.

Étienne câla son épaule contre la hanche de sa grand-mère
et resta quelques minutes à appuyer son poids contre elle.
%
La grande tempête avait aussi emporté sa mère.

-- Viens, si tu veux voir quelque chose d'utile, je t'amène à la fonderie.

Ce jour-là, sa grand-mère avait commencé à lui enseigner l'art sacré des
machines: les poulies, les leviers, les rails et les
engrenages que la fonderie produisait (les préférés d'Étienne).
%
Il s'était avéré un élève passionné.
%
Bientôt, il construisait ses propres machines et contribuait à réparer et
embellir celles du village.  
%
Il adorait chantonner les comptines qui guidaient
l'assemblage des pièces.  
%
Il adorait ajouter les volutes et les fioritures aux
moules servant à couler les engrenages. 
%
Sa grand-mère avait bien raison,
les machines étaient sublimes.  
%
Elle n'avait pas besoin d'être utiles, même si
plusieurs l'étaient.  
%
Elles étaient conçues en chantant et elles chantaient à leur tour.  
%
Elles rendaient la vie facile. 
%
Elles rendaient la
vie belle.

Partout sur la Terre, des enfants comme Étienne devenaient des bâtisseurs.

\sautSection{}

Tout le village était réuni pour un banquet. Chanson, histoire, musique, blah blah.

Information d'autres villages, chanson à traduire vu que ça vient d'autres langues.
Le décodage pour fabriquer une machine est aussi ce qui permet de décoder l'autre langue.

Histoires venant d'autres villages. Chanson et comptine pour encoder la fabrication. 
Tradition orale.

Étienne écoutait la comptine avec intérêt. Dans sa tête, il voyait comment combiner des pièces d'une
façon qu'il n'avait jamais pensé: une poulie ici, un engrenage à sept branche là. Oui, oui, ça fonctionne.
Il berçait sa tête au rythme de la comptine. Fermant les yeux, il tâchait de la mémoriser.
Demain, il pourrait constuire une nouvelle machine.

Prophétie de la dernière grande tempête. Nous sommes prêts.
Après cette dernière grande tempête, nous les humains allons quitter la Terre et jamais n'y revenir.

\sautSection{}

Étienne 

\sautSection{}

Village 

\sautSection{}

Étienne

\sautSection{}

Village. Terrible accident à la mine. Découverte d'une brique.
Une des machines d'Étienne avait déraillé.

Plusieurs blessés, couchés, avec de la fièvre: la tempête intérieure.

Dont sa femme, la mère de deux enfants.

-- Je peux aider. Mais leur corps va décider. Si c'est la fin.


Il connaissait le chatîment.

\sautSection{}

Tempête. Se réfugie dans les grottes. Beaucoup plus simple que le village, mais très très belle.
Beaucoup de nourriture. Des réserves d'eau. Un abri pouvant servir longtemps.


On vit dans le regard des anciens que cet ouragan n'était pas comme les autres.
Et pourquoi donc le ciel s'était-il illuminé de cette lumière orangé. En pleine nuit.

-- Si le temps est venu pour cette dernière tempête, nous sommes prêts.

Étienne jette un coup d'oeil à sa soeur. Les deux regardèrent le ciel s'embraser.
Il n'y avait aucun doute, c'était la dernière grande tempête.

Étienne regarda sa soeur. Son fils câla son épaule contre sa hanche et appuya son poids contre lui.

Une minute plus tard le ciel se mit à rougir.  
%
Ils eurent à peine le temps de
sentir un peu de chaleur pénétrer dans la grotte. 
%
L'air devint du feu et les
corps devinrent de la cendre.

Partout sur Terre, baignés d'amour et de beauté, chantant d'innombrables mélodies dans des milliers de langues,
les humains s'éteignirent.

%-- \ldots{} la fin n'est qu'une \ldots{} 

