\chapter{Harmonie}

Sur la place du village trônait une immense fresque mécanique, actionnée par
des petits moulins à vent.  Selon la force et la direction du vent, les 23
personnages de la fresque (Étienne les avait compté) changaient de place, se
déplaçaient sur des rails différents, aggripaient des outils différents.  À
chaque jour on pouvait voir une histoire différente... et entendre une mélodie
différente!  Une douce musique, un cliquetis de bois sur bois, mais calibré,
produisant des mélodies et des harmonies.

Étienne avait fait rire sa grand-mère le jour où il avait demandé à quoi la fresque pouvait bien servir.

-- À quoi ça sert? Tu es comme moi Étienne. La fresque n'a pas besoin de servir
à quelque chose. Elle est là pour t'émerveiller.  Elle est là pour être belle.
Savais-tu que cette fresque est à peine plus vieille que toi. On l'a bâtie en
arrivant ici, alors qu'une tempête avait détruit la dernière.  Si tu veux voir
quelque chose d'utile, vient je t'amène à la fonderie.







Tempête. Se réfugie dans les grottes. Encore très belles.


