\chapter{Harmonie}

Sur la place du village trônait une immense fresque mécanique, actionnée par
plusieurs petits moulins à vent.
%
Selon la force et la direction du vent, les 23 personnages de la fresque
(Étienne les avait compté) changaient de place, se déplaçaient sur des rails
différents, s'aggripaient à d'autres engrenages.  
%
À chaque jour on pouvait voir une autre histoire... et entendre une nouvelle
mélodie!  
%
Une douce musique, un cliquetis de bois sur bois où chaque pièce
avait été calibré.

Étienne avait fait rire sa grand-mère le jour où il avait demandé à quoi la
fresque pouvait bien servir.

-- À quoi ça sert? Tu es bien comme moi Étienne. La fresque n'a pas besoin de
servir à quelque chose. Elle est là pour émerveiller. Elle est là pour être
belle. Savais-tu que cette fresque est à peine plus vieille que toi. On l'a
bâtie en arrivant ici. La grande tempête avait emportré la dernière.

Les deux regardèrent et écoutèrent la fresque quelque temps sans rien dire. Étienne
se rapprocha de sa grand-mère et lui . La grande tempête avait aussi emporter sa mère.

-- Viens, si tu veux voir quelque chose d'utile, je t'amène à la fonderie.

Ce jour-là, sa grand-mère avait commencé à lui enseigner l'art sacré des
machines: les poulies, les leviers et, les préférés d'Étienne, les engrenages
que produisait la fonderie.



Tempête. Se réfugie dans les grottes. Encore très belles.

