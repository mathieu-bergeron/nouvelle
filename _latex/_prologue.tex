

Comme d'ailleurs, il n'y a personne pour poser la question: «le climat est-il en train de se stabiliser?». 

Comme vous l'avez déjà compris, les humains ne sont pas moins curieux qu'avant,
mais il n'existe plus personne possédant le bagage de connaissances minimal
pour arriver à formuler une telle question (et, évidemment, personne ne
comprendrait la réponse).  Les machines, qui détiennent toute cette richesse de
connaissances, ne prendront jamais l'initiative de la transmettre. L'humanité
est entré dans ce qu'on nomme en physique un régime permanent: un mouvement qui
ne va nulle part et qui, à moins d'un apport extérieur, ne sera jamais
perturbé.


    Dans les pays relativement épargnés par la crise du climat, 
    la Solution s'est déployée tout autrement.
    Les grandes entreprises ont
    

    Mais il ne s'était pas trompé sur l'efficacité de leur technologie.  Et
    malgré la pire introduction qu'on puisse imaginer pour une nouvelle
    technologie, les BLOCS se sont propagés.  Et la clé de leur succés
    résidaient dans leur capacité à s'adpater.  Pour certains, les BLOCS
    permettaient de simplifier une existance de préviligiés.  Pour d'autre, il
    s'agissiat d'un incroyable 


    Et ce n'était pas que des éléments de base. Chaque village possédait
    maintenant une petite fondrie capable de produire des microprocesseurs et
    autres composants électroniques, souvent à partir de matériaux recyclés ou
    troqués avec les villages voisins. Mais personne au village avait besoin
    d'y réfléchir ou même de le réaliser. Les instructions pour utliser les
    briques étaient simple. Bâtir sa fondrie était aussi simple que bâtir
    des abris. Puis le mur donnaient des instructions pour installer
    des senseurs ici et là sur le territoire. Puis le Mur donnait des
    instructions pour enseigner ceci et cela. Le matériel pédagogique était
    excellent et adapté à la langue et la culture locale.  Même le plan de
    développement était adapté.  Le Mur créait des clôtures dans les sociétés
    où c'était la norme et encourageait le nomadisme ailleur.

    La Solution était pensé pour maximiser l'autonomie de chaque village. 
    Après tout, les chemins commerciaux pouvaient être coupés à tout moment.
    Néanmoins, la Solution investissait beaucoup d'énergie à garder les liens
    de communication ouverts.

    Physiquement, on avait fabriqué un système nerveux à la planète. Un immense
    cerveau artificiel, capable d'assurer sa survie en utilisant les humains
    pour constuire et réparer son infrastructure (un brique par ci, un tuyau
    par là).  Culturellement, l'humanité s'était payé un surmoi algorithmique,
    capable de freiner nos ardeurs consuméristes et extractivistes. Capable de
    contrebalancer nos pulsions de violence avec juste la bonne idée au bon
    moment, juste la bonne perversion de la tradition narrative, juste le bon
    mensonge.

    La Machine se déclinait bien différemment d'un endroit à l'autre.  Au Sud,
    la Machine se voulait émancipatrice. Au Nord, elle visait à «rendre la vie
    plus simple».

    Dans les populations plus riches, la Machine se déclinait en gadget de
    toutes sortes, tous visant à «rendre la vie plus simple».


    Des appareils, cette fois-ci très jolis, l'ultime badge d'honneur pour pour
    la gauche champagne. Qui 


    Il est vrai que \nomProjet{} n'était pas très invitante, malgré les appels
    incessants des concepteurs visuels («si c'est laid, les gens ne vont pas
    l'utiliser!»).

    L'application s'adaptait aux différentes culture et traditions.
    


    C'est d'ailleurs pour cette même
    raison qu'on a su rapidement que l'IA aller jouer un grand rôle: puisque le
    problème était grandement lié au comportement humain, il nous fallait des
    solutions capablent d'influencer en profondeur le comportement humain. Et,
    bien que peu osait le formuler explicitement, c'était là la spécialité de
    l'informatique: prendre les décisions à la place des humains, et d'une
    façon où l'humain se croit encore en contrôle.



À l'intérieur du projet, il est rapidement apparu clair que l'informatique et
    l'intelligence artificielle allait joué un rôle crucial dans le projet.
    Pour coordonner un tel effort, bien sûr, mais aussi parce que la science
    était à l'époque un hybride entre intelligence humaine et artificielle.  Et
    pour une troisième raison, plus subtile: pour régler la crise du climat de
    façon durable, il fallait modifier le comportement des humains et, déjà à
    l'époque, l'informatique était la meilleure façon d'y arriver. 


    C'est ainsi
    qu'un large volet de \nomProjet{} s'est concentré à produire des
    technologies pour l'éducation des humains, dès la petite enfance jusqu'à la
    puberté. Ces technologies étaient conçues d'une façon à faciliter leur
    adoptions: les humains avaient toujours l'impression de choisir comment
    elles et ils les utilisaient. L'adoption été rapide.

    La machine était chargée d'installé chez les humains les connaissances et
    les savoirs-faires minimaux pour assurer la survie. Ce qui voulait souvent
    dire une littératie minimale (pour comprendre les instructions de la
    machine) et une copieuse diète de mythologie.

    On connaît exactement la suite et, bien que plusieurs penseurs ont aperçu à
    temps la débâcle, aucun n'a pu l'empêcher. 

    Puis il s'est passé quelque chose que personne n'avait prévu: le temps a
    passé sans avoir besoin de modifier les solutions proposés. L'IA s'adaptait
    suffissemment bien. Les rencontres du comité d'experts se succédaient sans
    propositions concrètes. L'intérêt pour ces comités ses lentement effrité.
    En fait, l'intérêt pour n'importe quel champ d'étude très poussé s'est
    effrité. Puis, l'intérêt pour d'autres disciplines. Sans même s'en rendre
    compte, sans même y penser, l'humanité était en train de détricoter des
    savoirs et des sciences qu'elle avait mis des millénaires à échafauder. 
    Or
    le cerveau humain n'est pas si costaud une fois qu'on lui a retiré tout
    l'apparaillage de connaissances et de savoir-faires que l'éducation a pour
    but d'installer.


Force est d'avouer que ce qu'est ce que les technologues crées: des machines
    qui focent les humains à s'adapter à comment la machine fonctionne, a
    travailler d'une certaine façon selon ce que la machine dicte, à penser
    comme la machine. Et les informaticiens en particulier, sont passé maître
    dans l'art de rendre cette subjugation invisible, de laisser croire aux
    humains qu'ils sont en contrôle.

La Machine n'a pas appris à rendre l'humain, mais a appris avec une efficacité
redoutable a rendre l'humain obéissant.

Ces technologies ont été rapidement adopté, par enthousiasme pour certain et
par nécessité pour d'autres (il y eut bien sûr des poches de résistance, qui se
sont rapidement estompé). Au début tout allait pour le mieux: les machines
prenaient en charge l'éducation des enfants en bas âge et l'éducation
supérieure continuait d'être administré par les humains.

Nous sommes au \siecle. L'humanité en est là: aucune capacité
d'adaptation. Tout dépend de la machine 
modifié (et plus personne ne sait même où commencer pour apprendre la science
nécessaire à la comprendre).
En fait, aucun humain n'a d'éducation supérieure à l'équivalent du primaire. Personne n'est particulièrement épanoui. La civilisation humaine a survécu, dans une version extrêmement amoindrie et fragilisée. Le prochain événement d'importance va automatiquement plonger l'humanité dans le retour à la préhistoire..
Remarquez, il ne reste plus personne capables d'intifier ce risque.

Certaines traditions se sont éteintes. Vrai, le dernier moteur à pétrole s'est éteint le 2301, mais aussi la dernière bouteille de vin a été bue le 2120. La culture humaine n'est pas éternelle et il est vraiment possible de perdre une technologie: imaginez que personne n'enseigne aux enfants ce qu'est le vin et comment le faire. Imaginer une société où l'initiative individuelle est savemment canalisé à des tâches sans conséquences.




Non, la fiction c'est moi. C'est ce Je qui vous parle. Nous sommes au \siecle{} et il existe aucune âme sur Terre capable de raconter ce que je vous ai raconté. Les faits sont bien enregistrés, et une machine serait sans aucun doute capable de générer ce texte, mais aucun humain n'en aurait l'idée et les machines sont sciemment conçues pour ne pas prendre ce genre d'initiative.

Ce personnage est bien réel. C'est moi, le narrateur, qui est une fiction. Je suis une fiction parce qu'il ne reste plus personne, au \siecle{}, qui pourrait vous expliquer l'histoire de l'humanité comme je l'ai fait. Parmi les machines, qui elles pourraient générer le texte que vous lisez, aucune n'a la capacité de prendre une telle initiative.


    Mais la technologie avait été bien pensé: modulaire, réparable, recyclable,
    utilisant l'énergie solaire dès que possible, rarement en panne, etc.
