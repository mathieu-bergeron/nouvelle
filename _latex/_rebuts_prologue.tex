


On ne saurait jamais combien de spectacteurs ont réalisé que la majorité des représentants 
était en fait des doublure. Mesure de sécurité obligatoire.

Barkisu par exemple -- la vraie Barkisu -- était déjà dans son labo sur une
ancienne plate-forme pétrolière reconvertie en centre de recherche flottant.




On avait ratissé large.


Derrière le rideau, on savait très bien que le projet \nomProjet{} était 

oscillait entre le chaos et la lourdeur bureaucratique et diplomatique.

Difficile à organiser (qui sait, peut-être avait-on vraiment recruté quelqu'un se disant expert de la physique shamanique).

Il aura fallu dix ans pour mettre le projet en branle.
Il aura fallu des années de négociations pour déterminer
Il aura fallu des mois de négociation pour se mettre d'accord sur le nom officiel du projet, sur qui allait prononcer les première paroles
lors de la cérémonie d'ouverture (et dans quelle langue).

Derrière les rideaux, on s'inquiétait beaucoup.
À ce rythme, avec cette complexité, on arriverait à rien.

beaucoup plus c

Pourtant \nomProjet{} n'était pas 



Malgré les railleries: un seul objectif, plutôt \og à la recherche d'un objectif \fg.





Le clou du spectacle était la présentation de chaque délégation.



Voir des intellectuels déambulés dans un stade aurait fait rigoler, mais pour
plusieurs. Voici le contingent américains. On reconnaît l'éminent.
Voici le contingent canadian. 
On reconnaît Blah blhi de la nation Blah blah blah.

En d'autres circonstances, 
voir des intellectuels déambul

Des années à discuter, des années à négocier le moindre détail. 
Des années à discuter du nom projet. Des centaines de chroniques
sur la prononciation exacte de \nomProjet{} (le \og ù \fg doit se prononcer en glissant vers le bas).

Tel était l'humanité à l'époque.

Très difficile à organiser.  Opposition, manifestation. Critique que pendant ce
temps, on mourait dans les camps de réfugiés.
Piratage informatique.


Dans le siècle à venir, à travers les cataclysmes et les conflits, l'humanité
va démontrer quelque chose d'innatendu: les grandes puissances annonceront une
alliance surprise et débloqueront des fonds sans précédent pour un projet nommé
\textit{\nomProjet{}} (\nomProjetEn{}), un projet Manhattan multinational, un
effort titanesque visant à se débarrasser une fois pour toute de la menace
climatique et, de façon plus urgente, à éviter une enième guerre mondiale.  Les
nations plus petites se rallieront sans hésitation.  Tout ce qui restera de
force vive s'engagera pour la cause: survivre.

Le projet \nomProjet{} se déploiera à une vitesse fulgurante.  Toutes les
formes de connaissance y seront bienvenue, tout ce qui pourrait avoir le
pouvoir de construire une nouvelle humanité.  les savoirs qui ont eu le pouvoir
de traverser les époques, les savoirs tout neuf qui peut-être débloquent de
nouvelles portes Tous ce qui est capable de capturer l'imaginaire de
l'humanité, de se préserver malgré les tempêtes.



recrutant parmi une vaste fourchette d'expertises.  
Toutes formes de connaissance sera la bienvenu:
