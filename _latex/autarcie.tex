\chapter{Autarcie}

\section*{Question}

\noindent\colorbox{gray!30}{\framebox{\parbox{\textwidth}{\texttt{Comment l'humanité a-t-elle survécu aux grandes crises climatiques?}}}}\\

\section*{Réponse}

Je vais d'abord récapituler comment la solution proposée par le projet
\nomProjet{} a été conçue et implantée.  Mais pour bien expliquer comment cette
solution fonctionne, et quels en sont les impacts, je préfère m'en remettre à
la fiction.  il sera préférable je crois de vous l'illustrer à travers la vie
de trois humains qui ont réellement existé. Ces humains n'ont rien de bien
spécial, il sont au contraire représentatif de l'humanité telle que nous la
conaissons aujourd'hui.


\paragraph{Rebâtir le monde, brique par brique}

Nous sommes au \siecle. La vie n'est pas si mal pour les humains, même si le
climat commence à peine à se rétablir. Après quatre cents ans! Quatre cents ans à
s'abriter des tornades, à endurer les sécheresses, à fuir les inondations, à se
protéger du froid, de la chaleur, des tempêtes incessantes. Quatre cents ans
sans décider quand sortir de son foyer, ni où aller, ni avec qui.
Quatre cents ans à survire.

Pour vous expliquer comment l'humanité a su se transformer, j'aimerais
laisser la parole à \nomMere{}, 32 ans, et à son fils \nomEnfant{}, 8 ans.
Mais pour rendre ma fiction utile, je dois d'abord faire un peu
d'histoire.  


Le projet \nomProjet{} s'est déployé à une vi\-tesse
fulgurante, recrutant parmi une vaste fourchette d'expertises.  On y
retrouvait des autochtones, venus partager leur connaissance intime de la
terre, guidant des sommités mathé\-matiques dans leur modélisation du climat et
leurs formalisations de stra\-tégies permettant de s'adapter tant
bien que mal.  On y retrouvait des artistes, venus partager leur talent à
émouvoir et inspirer l'humain, développant avec des psychologues des
tactiques visant à vaincre l'in\-act\-ion et l'apathie.  On y trouvait des
philosophes, papillonnant d'un projet à l'autre, tentant d'identifier, à
l'avance, les culs-de-sac con\-cept\-uels de tout un chacun.  On y trouvait
des économistes qui, laissant de côté leurs chicanes habituelles,
réfléchissaient aux moyens de rendre les solutions proposées viables à
large échelle et à long terme.  On y trouvait aussi, bien sûr, un grand
nombre d'ingénieurs et autres technologues qui, comme c'était trist\-ement
le cas à l'époque, étaient enclins (ou contraints) à promouvoir les
intérêts de leurs employeurs, des grandes entreprises, de la finance.

D'ailleurs, le projet \nomProjet{} comportait sans surprise tous les
travers de son époque: guerres intestines, violences de toute sorte, débats
stériles alimentés par les médias, propositions
subti\-le\-ment sexistes ou racistes (et parfois pas subtiles du tout), et
j'en passes.

À l'extérieur du projet, on rêvait de grands chantiers.  On choisissait
déjà où installer les usines gigantesques qui allaient assainir et
contrô\-ler l'at\-mosphère de la Terre. On se demandait déjà de quoi la vie
aurait l'air à bord d'une station spatiale, ou encore à l'intérieur d'une
mégapole souterraine. On se préparait déjà à accepter la création d'une
nouvelle espèce humaine, génétiquement modifiée afin de respirer du gaz
carbo\-nique.

À l'intérieur du projet, on a tout de suite compris que ces grands chantiers
n'auraient jamais lieu: on désespérait à coordonner nos efforts! La
communication était pénible, les équipes trop disparates. Pire, personne ne
trouvait comment recombiner les multiples avancées du projet en un tout
cohérent.  Le problème était si envahissant qu'après quelques années à peine,
le projet \nomProjet{}, l'espoir de l'humanité, en était réduit à gaspiller la
majeure partie de ses énergies à se maintenir organisé, à éviter de s'effondrer
de l'intérieur.  

Comme c'était son habitude, l'informatique a saisi l'occasion
pour s'immiscer partout, à grand coup d'assistants virtuels et d'intelligence
artificielle (IA).  Et à la surprise de tous (sauf des informati\-ciennes),
c'est de cette informatique, perçue à tort comme un simple outil, qu'à
émergé la Solution.

Une Solution que l'humanité a d'abord dénoncée, décriée, conspuée: «Une
application? Des appareils électroniques? Nous on meure et vous, vous créez des
jouets?» rageait-on en crachant au sol, en montrant le poing, en écrasant
les touches du clavier.  Même le design des appareils provoquait le
dégoût: des genres de briques grises qui s'emboîtaient mala\-droi\-tement, ou
encore des tuyaux beiges dans lesquels s'engouffraient des touffes
de câbles.

%La Solution n'était pas tant une solution que des technologies allant 
%permettre l'émergence d'une solution.

Mais les gens de \nomProjet{} ne s'étaient pas trompé sur la portée de
leur technologie.
Les briques ont d'abord été envoyées dans les
camps de réfugiés climatiques qui pullu\-laient un peu partout sur Terre.
L'emboît\-e\-ment des briques, en apparence si maladroit,
était en fait légèrement robotisé: les briques changeaient subtilement de
forme et s'a\-grip\-paient les unes aux autres. En moins d'une semaine, on
pouvait transfor\-mer un camp en petit village.

Et il y a plus. Quand on empilait douze briques pour former un bloc, ce
dernier se scellait hermétiquement et se réorganisait à l'intérieur. Ces
blocs pouvaient ensuite servir à construire quelque chose de plus gros.
D'ailleurs, dans les camps devenus villages, on
s'empressait à construire une \textit{fonderie}: une usine automatisée qui
pouvait fabriquer de nouvelles briques.

La vie dans les camps demeurait néanmoins très rude. On creusait pour
enfouir les tuyaux et les câbles qui reliaient chaque bâtiment. On creusait
pour enfouir les égouts.  On mangeait la bouillie immonde que la cantine
automatisée produisait (à partir de tout ce qu'on pouvait dénicher de
matière organique).  On s'aven\-turait dans les tempêtes, les déserts ou
les villes inondées, à la recherche de matériaux pour alimenter la
fonderie.  On fouillait les ruines et les déchets de l'ancien monde pour en
construire un nouveau, beaucoup plus laid.
On survivait.

En harnachant cette force de vivre, les camps devenaient
des villes autosuffisantes, puis des petits centres industriels, capables
de fournir en briques toute une région. Capables de propager la
Solution.

Une Solution qui, comme vous vous en doutez, se déclinait bien autrement
parmi les populations riches, qui avaient pu protéger leur
mode de vie.  Chez les riches, on a tout de suite méprisé les briques (une
technologie pour le tiers-monde!). L'application \nomProjet{}, par contre,
y a connu un essor immédiat, malgré le barrage de commentaires négatifs (ou,
peut-être, grâce à la curiosité provoquée par ce vitriol).

À première vue, l'application n'offrait rien de plus que
les systèmes informatiques de l'époque. C'était à se demander ce
qu'on espérait accomplir en fournis\-sant aux humains une énième façon de
communiquer, de se renseigner, de déléguer certaines tâches lassantes à une
intelligence artificielle qui, jusqu'à ce jour, n'avait jamais remplie sa
promesse: rendre la vie plus simple.

Mais les gens de \nomProjet{} ne s'étaient pas trompé sur la portée de leur
technologie. Leur IA avait une capacité d'apprentissage et d'adapt\-ation
largement supérieure à tout ce que l'humanité avait connu jusqu'alors. Elle
épousait toutes les langues, toutes les cultures, tous les systèmes de pensée.
Elle avait le don de parler à chaque humain à travers ses mythes préférés.
Elle avait le don de mettre la bonne information devant les bons yeux, et au
bon moment! 

Pour décourager la consommation de viande, l'IA pouvait autant se référer à
l'ahimsâ, que souligner un prix trop élevé; autant évoquer la souffrance des
animaux, que les risques de salmonellose.  Et il en était de même pour une
foule d'autres sujets. L'IA entrete\-nait des contacts fréquents avec la
quasi-totalité des internautes (d'autant plus qu'elle avait été
dis\-crè\-tement implantée dans plusieurs applications).  En moins d'un an,
elle en avait appris suffisam\-ment sur l'humain pour arriver à découper les
internautes en sous-groupes assez nets.  Comme les briques qui devenaient des
blocs, elle créait des com\-munautés, les scellait herméti\-que\-ment et les
réorganisait de l'intérieur.

Usant de son pouvoir d'influence, l'IA a ensuite piloté l'introduction
des briques partout dans le monde, y compris dans les régions plus riches,
comme elle avait été programmée pour le faire. Car
les briques et l'IA fonctionnaient en symbiose. Il s'agissait du \textit{tout
cohérent} que le projet \nomProjet{} avait eu tant de mal à imaginer.
Voyez-vous, les fonderies ne s'appelaient pas comme ça par hasard: elles
fabri\-quaient, cachés dans leurs briques, des petits ordinateurs.  On pouvait
d'ail\-leurs y apposer des écrans ou des haut-parleurs. Avec les briques, nul
besoin de télécharger l'application \nomProjet{}, elle était intégrée dans les
murs!  Ces murs fournissaient la puissance de calcul requise pour faire vivre
l'IA, mais aussi la masse de données dont elle avait besoin pour se parfaire.
L'IA avait gagné la possibilité de se reproduire, de croître, de s'immiscer
partout.

%Mais il y a plus. La capacité d'anticipation de l'IA était sans
%commune mesure. Ses prévisions météos étaient rigoureusement exactes.  Elle
%manipulait le marché pour amortir les impacts négatif de ses décisions.
%minimi En poussant les producteurs L'IA prévoyait même l'impact de ses
%décisions et amortissait leurs effets négatifs. Les producteurs de viandes
%Et ce 40\% de diminution, est-ce que ça a entraîné une crise économique. 
%Non, l'IA conseillait aussi les producteurs, a changer de gagne pain.

%Et ça fonctionnait! Rapidement, on a mesuré l'impact de l'IA sur l'empreinte
%écologique des humains.
%Déjà, on mesurait
%l'impact sur l'empreinte écologique des humains: les vols avaient
%diminués du quart, la taille du parc automobile stagnait et la consommation
%de viande commençait à chuter.

%L'IA, qui avait à l'interne une notion très forte de vérité (elle devait
%après tout prévoir un paquet de choses, dont la météo), ne trouvait
%apparemment pas utile de partager cette notion les humains.

%Le projet \nomProjet{} s'est dissout. Il ne restait qu'un
%comité d'experts chargé d'appliquer des modifications mineures au système.





\paragraph{Adèle, la coureuses des bois}

\paragraph{Gregson, le bon père} 

\paragraph{Amélie, l'hédoniste}

Bon, voilà, je pense que j'ai fait assez d'histoire. Je peux enfin laisser la
parole à \nomMere, 32 ans, mère de \nomEnfant, 8 ans. Ou, en fait, à des
extraits de leurs données. Vous allez comprendre.


\noindent\begin{longtable}{rl}
$[$ \textit{Action requise} $]$ & la brique 420 doit être changée~~\colorbox{gray!30}{\framebox{\textsc{ok}}}~\framebox{\textsc{plus tard}}\\
$[$ \nomMere{} $]$ & La 420? C'est où ça? C'est dans la toilette, non?\\
$[$ \nomMere{} $]$ & Jadon! Viens mon c\oe{}ur, je veux te montrer quelque chose\\
$[$ \nomMere{} $]$ & Jadon!\\
\ldots{}~~~\\
$[$ \textit{Action requise} $]$ & Paquet à acheminer (3kg)~~\framebox{\textsc{ok}}~\colorbox{gray!30}{\framebox{\textsc{plus tard}}}\\
\ldots{}~~~\\
$[$ \nomEnfant{} $]$ & Maman, est-ce que ça existe pour vrai les poissons?\\
$[$ \nomMere{} $]$ & Où as-tu entendu ça mon c\oe{}ur?\\
$[$ \nomEnfant{} $]$ & Dans mon émission scolaire, il fallait en dessiner\\
$[$ \nomMere{} $]$ & Sais-tu, je suis pas sûre\\
$[$ \nomMere{} $]$ & Mais je peux t'aider si tu veux\\
$[$ \nomMere{} $]$ & J'aimerais bien les dessiner avec toi\\
\ldots{}~~~\\
$[$ \textit{Action requise} $]$ & La brique 112 doit être changée~~\colorbox{gray!30}{\framebox{\textsc{ok}}}~\framebox{\textsc{plus tard}}\\
%$[\ldots]$\\
%$[$ \nomEnfant{} $]$ Maman, est-ce qu'on peut aller jouer dehors?\\
%$[$ \nomMere{} $]$ Je sais pas, ça sent la tempête aujourd'hui\\
%$[$ \textit{Prévisions météo} $]$ Prochaine accalmie dans 2 jours~~\colorbox{gray!30}{\framebox{\textsc{ok}}}\\
%\ldots{}~~~\\

\end{longtable}

\paragraph{À propos de l'usage de fiction}

Je suis devenu convaincu que cette réponse ne peux exister ailleurs que 
dans la fiction.

On peut s'arrêter ici. Le reste des données est similaire.  En fait, les
données de tous les humains sont similaires. Pas en surface, bien sûr.  En surface,
il y a une énorme diversité de langues, de cultures, de passe-temps, de façons
de se parler et de s'aimer. Mais dans le fond, croyez-moi, ces données
finissent toutes par se ressembler.  La vie des humains, au \siecle{}, est
finalement simple.

Je pense que vous comprenez maintenant et je pense que j'ai répondu
adéquatement à la question «comment l'humanité a-t-elle traversé la grande
crise climatique?».  Mais c'est important pour moi de clarifier quelque chose.
Vous vous souvenez peut-être que j'ai utilisé l'expression \textit{fiction
utile} au début du texte.  Je veux clarifier que \nomMere{} a réellement existé.
Elle est née le 22 janvier 2321 dans une salle d'opération automatisée, à
Matagami. Elle a donné naissance à son fils au même endroit, le 4 août 2345, et
elle est décédé à l'âge de 83 ans, dans son lit, entourée de
son fils et de sa petite-fille.

La fiction ici, chère lectrice ou cher lecteur, c'est vous. Ce texte existe
pour vrai.  Je l'ai rédigé à l'avance, comme je me dois de le faire, en
prévision d'une question qui, jusqu'à ce jour, ne m'a pas été posée.  Et bien
que je n'aie aucune modification concrète à apporter à ce texte, je le révise
en continu, comme il se doit, dès que j'ai un peu de temps libre.  J'insiste.
La fiction ici, c'est vous.  Je peux vous l'affirmer, moi qui connais les
moindres recoins de la Terre: l'humanité ne possède plus les
connaissances de base qui permettrait d'imaginer une question de cette ampleur.
Pour les réintroduire, il faudrait d'abord réintroduire l'écriture, et ce n'est
pas là une mince affaire! Non. Ces connaissances de base, et tant d'autres encore,
je me garde bien de les partager.  Comme il se doit. Comme on m'a programmé
pour le faire.

Nous sommes au \siecle{}. Le projet \nomProjet{} a accompli son objectif. L'espèce humaine a survécu.
