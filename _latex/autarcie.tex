\chapter{Autarcie}

\section*{Question}

\noindent\colorbox{gray!30}{\framebox{\parbox{\textwidth}{\texttt{Comment l'humanité a-t-elle survécu aux grandes crises climatiques?}}}}\\

\section*{Réponse}


C'est une question qui demande tellement d'explications que je préfère y
répondre en te racontant quatre fables. Il s'agit de fables que tu connais bien,
que je te reconte depuis ton enfance. 

Une fable pour récapituler comment le monde s'est formé, puis trois fables pour
te faire apprécier comment les humains peuvent vivre des vies bien différentes
de la tienne.

Je vais te raconter les mêmes fables que lorsque tu étais enfant.  Et je vais
former sur le mur les mêmes figurines animées.  Mais écoute bien le son de ma
voix, écoutes bien mes intonations.  Je vais souligner certains mots,
\textit{comme ceci}, pour te laisser deviner que l'histoire que je reconte aux
enfants n'est pas tout à fait complète.  Pour chaque mot souligné, tu pourras
lire les annotations ci-bas, voire me demander plus de détails.

\colorbox{gray!30}{\framebox{Lire les annotations ici}}

Je vais d'abord récapituler comment la solution proposée par le projet
\nomProjet{} a été conçue et implantée.  Mais pour bien expliquer comment cette
solution fonctionne, et quels en sont les impacts, je préfère m'en remettre à
la fiction.  il sera préférable je crois de vous l'illustrer à travers la vie
de trois humains qui ont réellement existé. Ces humains n'ont rien de bien
spécial, il sont au contraire représentatif de l'humanité telle que nous la
conaissons aujourd'hui.

Je veux aussi que tu portes attention à comment je vais te raconter ces histoires.
Assis-toi devant le mur. Remarques bien comment les briques bougent.
Écoute bien le son de ma voix.
Rien de bien différent à ce que tu as vu des centaines et des centaines de fois.
Mais portes bien attention aux annotations.

\paragraph{La fin des grandes crises}

Ma première fable vise à récapituler comment le monde que tu connais s'est formé.


\paragraph{Adèle, la coureuses des bois}

\paragraph{Gregson, le bon père} 

\paragraph{Amélie, l'hédoniste}

Bon, voilà, je pense que j'ai fait assez d'histoire. Je peux enfin laisser la
parole à \nomMere, 32 ans, mère de \nomEnfant, 8 ans. Ou, en fait, à des
extraits de leurs données. Vous allez comprendre.


\noindent\begin{longtable}{rl}
$[$ \textit{Action requise} $]$ & la brique 420 doit être changée~~\colorbox{gray!30}{\framebox{\textsc{ok}}}~\framebox{\textsc{plus tard}}\\
$[$ \nomMere{} $]$ & La 420? C'est où ça? C'est dans la toilette, non?\\
$[$ \nomMere{} $]$ & Jadon! Viens mon c\oe{}ur, je veux te montrer quelque chose\\
$[$ \nomMere{} $]$ & Jadon!\\
\ldots{}~~~\\
$[$ \textit{Action requise} $]$ & Paquet à acheminer (3kg)~~\framebox{\textsc{ok}}~\colorbox{gray!30}{\framebox{\textsc{plus tard}}}\\
\ldots{}~~~\\
$[$ \nomEnfant{} $]$ & Maman, est-ce que ça existe pour vrai les poissons?\\
$[$ \nomMere{} $]$ & Où as-tu entendu ça mon c\oe{}ur?\\
$[$ \nomEnfant{} $]$ & Dans mon émission scolaire, il fallait en dessiner\\
$[$ \nomMere{} $]$ & Sais-tu, je suis pas sûre\\
$[$ \nomMere{} $]$ & Mais je peux t'aider si tu veux\\
$[$ \nomMere{} $]$ & J'aimerais bien les dessiner avec toi\\
\ldots{}~~~\\
$[$ \textit{Action requise} $]$ & La brique 112 doit être changée~~\colorbox{gray!30}{\framebox{\textsc{ok}}}~\framebox{\textsc{plus tard}}\\
%$[\ldots]$\\
%$[$ \nomEnfant{} $]$ Maman, est-ce qu'on peut aller jouer dehors?\\
%$[$ \nomMere{} $]$ Je sais pas, ça sent la tempête aujourd'hui\\
%$[$ \textit{Prévisions météo} $]$ Prochaine accalmie dans 2 jours~~\colorbox{gray!30}{\framebox{\textsc{ok}}}\\
%\ldots{}~~~\\

\end{longtable}

\paragraph{À propos de l'usage de fiction}

Je suis devenu convaincu que cette réponse ne peux exister ailleurs que 
dans la fiction.

On peut s'arrêter ici. Le reste des données est similaire.  En fait, les
données de tous les humains sont similaires. Pas en surface, bien sûr.  En surface,
il y a une énorme diversité de langues, de cultures, de passe-temps, de façons
de se parler et de s'aimer. Mais dans le fond, croyez-moi, ces données
finissent toutes par se ressembler.  La vie des humains, au \siecle{}, est
finalement simple.

Je pense que vous comprenez maintenant et je pense que j'ai répondu
adéquatement à la question «comment l'humanité a-t-elle traversé la grande
crise climatique?».  Mais c'est important pour moi de clarifier quelque chose.
Vous vous souvenez peut-être que j'ai utilisé l'expression \textit{fiction
utile} au début du texte.  Je veux clarifier que \nomMere{} a réellement existé.
Elle est née le 22 janvier 2321 dans une salle d'opération automatisée, à
Matagami. Elle a donné naissance à son fils au même endroit, le 4 août 2345, et
elle est décédé à l'âge de 83 ans, dans son lit, entourée de
son fils et de sa petite-fille.

La fiction ici, chère lectrice ou cher lecteur, c'est vous. Ce texte existe
pour vrai.  Je l'ai rédigé à l'avance, comme je me dois de le faire, en
prévision d'une question qui, jusqu'à ce jour, ne m'a pas été posée.  Et bien
que je n'aie aucune modification concrète à apporter à ce texte, je le révise
en continu, comme il se doit, dès que j'ai un peu de temps libre.  J'insiste.
La fiction ici, c'est vous.  Je peux vous l'affirmer, moi qui connais les
moindres recoins de la Terre: l'humanité ne possède plus les
connaissances de base qui permettrait d'imaginer une question de cette ampleur.
Pour les réintroduire, il faudrait d'abord réintroduire l'écriture, et ce n'est
pas là une mince affaire! Non. Ces connaissances de base, et tant d'autres encore,
je me garde bien de les partager.  Comme il se doit. Comme on m'a programmé
pour le faire.

Nous sommes au \siecle{}. Le projet \nomProjet{} a accompli son objectif. L'espèce humaine a survécu.


FICHE RÉCAPITULATIVE: taux de littératie, taux de codératie, etc.
