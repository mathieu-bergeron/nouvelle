\chapter{Transition}

Barkisu guettait le bassin où Misha venait s'installer à tous les jours.
%
Elle avait toujours une pointe d'inquiétude en attendant que Misha apparaisse
dans le bassin de droite, comme elle faisait pourtant à chaque jour.  
%
Elle ne pouvait s'empêcher d'espérer que Yaya ne les rejoignent aussi, même
s'il n'était pas venu depuis des mois.
% 
Penser à Yaya la faisait souffrir, mais elle ne pouvait s'en empêcher.
%
Elle avait vraiment peur pour lui. 
%
Plus encore, elle avait peur parce qu'il était devenu violent. 
%
Un de ces jours, il va s'en prendre à Misha, c'est évident.

-- Bonjour mère.

-- Misha, je suis tellement contente! Es-tu bien? Trouves-tu à manger?

Selon le protocole de recherche qu'elle avait elle-même élaboré avec Djamila,
elle n'aurait jamais dû laisser deux spécimen atteindre la maturité (même si techniquement Yaya et Misha étaient plutôt des juvéniles).
%

-- C'est quelle génération aujourd'hui? La trois cent quarante-quatrième?

C'était un petit jeu qu'elle avait avec Misha, qui posait toujours cette
question tout en connaissant parfaitement la réponse.
%
Peut-être que Misha voulait véfifier que l'esprit de Bariksu était encore affuté.
%
C'est vrai qu'elle avait vraiment vieilli ces dernières années. 
%
%
Elle se sentait faible.
%
La journée où elle s'était engagé dans le projet \nomProjet{} semblait si loin.
% 
Elle pensait beaucoup à Djamila, sa collaboratrice, l'amour de sa vie.
%
Et elle était en colère.
%
Elle rageait contre la mort de Djamila, quel accident terrible!
%
Elle rageait toujours que \nomProjet{} ce soit écroulé aussi rapidement, aussi
spéctaculairement.
%
Elle préférait ne pas penser au continent, mais elle l'imaginait consumé par la
guerre.
%
Un jour des bâteaux emplis d'hommes violents approcheraient la plate-forme.
Elle le sentait, c'était une question de temps. 
Ils seraient à la recherche de quelque chose de précieux: de la nourriture, des briques encores fonctionnelles, n'importe
quoi.
%
La fin de son travail approchait. 
%
Bientôt, il lui faudrait déclarer la fin des générations cultivées et laisser la nature prendre son cours.
%

-- Tu sais très bien que c'est la trois *mille* quarante-quatrième mon amour. Dis-moi, l'espèce \textit{homo sapiens} est apparue il y a trois mille ans, c'est bien ça?

Un autre jeu. 
%
Cette fois Barkisu voulait vérifier que Misha avec lu les derniers articles qu'elle avait
encodés et lancer à la mer.
%

-- Ah oui, j'ai vu cet article! Trois mille ans ou trois cent mille ans, quelle différence?

-- Héhé. En passant, as-tu réfléchis à votre nom: \textit{homo piscis}, \textit{homo aquaticus}? As-tu une préférence?

-- Oui, j'ai réfléchis. Je n'aime pas du tout l'idée d'un nom d'espèce qui fasse référence à mes branchies. Ce n'est pas là ce qui me distingue à mon avis. 
   Si homo sapiens veut dire \textit{humain sage} (ce qui est un assez gros mensonge), j'aimerais
   proposer \textit{homo scribis}, l'humain qui écrit. J'aime bien. Et en plus c'est la vérité!

C'était en effet la vérité. Depuis la génération 185, les specimen savaient
lire et d'écrire de façon innée. C'était la contribution de Djamila à ce
projet. Une contribution immense.  Trois doigts produisaient des enzymes bien
distinctes (le pouce et l'auriculaire étaient neutre).  En manipulant un dé à
six faces, un \textit{scribis} pouvait encoder trois valeurs sur chaque face:
donc 3 à la 6 valeurs au total: 729 valeurs au total.  Avec un dé à 8 faces, on
monte à 6561 valeurs.  Djamila avait donc intégré un système d'écriture à même
les réflexes qu'ont les bébés de tout toucher et de tout mettre dans leur
bouche.  Si le dé était plat, les neufs mots représentés était: amour, danger,
parent, nourriture (ou appétit), encore, terminé.  Avec un dé à quatre faces
(81 symboles), on représentait les mêmes neuf mots, auquel on ajoutait 16 mots,
15 marques de ponctuation, dix chiffres, cinq symboles mathématiques et 26 symboles
«génériques» qui pourraient, par exemple, permettre d'utiliser un alphabet.
Dès l'enfance, un \textit{scribis} va lire (le réflexe est de recopier,
renforcer les symboles lus).  Selon les simulations, un \textit{scribis}
pouvait apprendre à lire et écrire de façon innée, contrairement au
\textit{sapiens}.  C'était tout de même préférable qu'un adulte les guide,
montrant à l'enfant à lire (savourer) et écrire (engluer) des dés à deux faces
d'abord, puis à trois, puis à quatre, etc.

Djamila avait aussi ré-inséré le même système dans les babiages.  Avec les
branchies, les \textit{scribis} ne pouvait pas former de phonème aussi clair
que les \textit{sapiens}.  Mais les bébés n'auraient aucune difficulté à
chantonner, à babier.  En marquant des pauses, on pouvait facilement distinguer
des séquences de deux ou quatres sons. On pouvait encoder les mêmes mots de
base. Deux trait aigu pour adulte, deux trait grave pour danger, un trait aigu
suivit d'un trait grave pour nourriture, etc.


-- Faudrait demander à ton frère ce qu'il en pense, mais j'aime assez.

Le terminal clignota pendant de longues minutes avant que Masha ne réponde. 

-- Maman. Tu sais que Yaya... tu sais qu'il ne me parle plus vraiment.

-- Oui je sais, je sais mon c\oe ur. Bon, on y va pour homo scribis alors. As-tu envie de travailler aujourd'hui?

Misha s'était avéré assez capable pour la génétique et le codage, mais ça ne lui paraissait pas très passionnant.
La tâche de choisir quels textes envoyés dans la mer, alors là!
Un peu de mécanique des fluides, essentiel.

Le plus clair de son temps était plutôt dévolu à lire goulument et à écrire.

Barkisu se sentait bien seule. Vieille et seule.
Djamila, sa collaboratrice, sa compagne, l'amour de sa vie, n'était plus.
Seule dans cette gigantesque plate-forme flottante, jadis une ville bourdonnante.

Elle était encore blessée, elle qui avait tellement cru au projet \nomProjet{},
qui avait pour ainsi dire grandit dans ce projet. C'était sa vie.

Elle imaginait le monde autour d'elle descendu dans la guerre.

Elle imaginait, mais me préférait pas savoir, préférait continuer le travail.
Le travail de sa vie. 

De désespoir et de solitude, elle laissa un spécimen de la 300ième génération atteindre la maturité.
Elle 

Jusqu'au jour où elle a 

Il y a avait Yaya, le fils qu'elle avait perdu.
Il ne restait que Alpha, l'enfant qui lui restait.

\sautSection{}

Hermaprhodite. Journée de grand stress (la mort de Yaya) se solde par Misha qui pond un oeuf déjà fécondé, ce qui est très différent comme expérience puisque Misha sera enceinte pendant quelques semaines avant de pondre l'oeuf (plutôt que quelques jours de malaises pour un oeuf non-fécondé: comme les menstruations).

\sautSection{}

Cadeau de Misha pour Barkisu: un texte sur l'histoire de l'espèce. Et Misha a bien suivi les conseils de Barkisu et n'a pas inclus d'individus. \og c'est trop risqué de créer une mythologie basée sur l'individu. Les faits c'est bien assez. 
Les scribis se créeront leur propre mythologie\ldots{} ou peut-être pas du tout, peut-être vous aurez uniquement de l'art \fg{}.

\sautSection{}

-- Et puis tu vas être occupé Misha, tu deviens le parent d'un millions de \textit{scribis}.

Sous la mer, un grand filet s'ouvrit. Des millions d'oeufs fécondés coulèrent
doucement vers les profondeurs de l'océan.


