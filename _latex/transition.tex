\chapter{Transition}

Barkisu alluma le terminal donnant sous la mer, un terminal immergé à près
vingt mètres de profondeur. 
%
Comme à chaque jour elle espérait que Misha vienne lui écrire quelque chose.  
%
Ou peut-être Yaya, même s'il n'avait pas écrit depuis des mois.  
% 
Encore aujourd'hui, penser à Yaya la faisait souffrir.  
%
Elle avait vraiment
peur pour lui. 
%
Plus encore, elle avait peur parce qu'il était devenu violent. 
%
Un de ces jours, il va s'en prendre à Misha, c'est évident.

-- Bonjour mère.

-- Misha, je suis tellement contente! Es-tu bien? Trouves-tu à manger?

Selon le protocole de recherche qu'elle avait elle-même élaboré avec Djamila,
elle n'aurait jamais dû laisser deux spécimen atteindre la maturité (même si techniquement Yaya et Misha étaient plutôt des juvéniles).
%

-- C'est quelle génération aujourd'hui? La 344?

C'était un petit jeu qu'elle avait avec Misha, qui posait toujours cette
question tout en connaissant parfaitement la réponse.
%
Peut-être que Misha voulait véfifier que l'esprit de Bariksu était encore affuté.
%
C'est vrai qu'elle avait vraiment vieilli ces dernières années. 
%
Elle se sentait faible.
% 
Elle pensait beaucoup à Djamila, sa collaboratrice, l'amour de sa vie.
%
Et elle était en colère.
%
Elle rageait contre la mort de Djamila, quel accident terrible!
%
Elle rageait toujours que \nomProjet{} ce soit écroulé aussi rapidement, aussi
spéctaculairement.
%
Elle préférait ne pas penser au continent, mais elle l'imaginait consumé par la
guerre.
%
Un jour des bâteaux emplis d'hommes violents approcheraient la plate-forme.
Elle le sentait, c'était une question de temps. 
Ils seraient à la recherche de quelque chose de précieux: de la nourriture, des briques encores fonctionnelles, n'importe
quoi.
%
La fin de son travail approchait. 
%
Bientôt, il lui faudrait déclarer la fin des générations cultivées et laisser la nature prendre son cours.
%

-- Tu sais très bien que c'est la 345 mon amour. Dis-moi, l'espèce \textit{homo sapiens} est apparue il y a trois mille ans, c'est bien ça?

Un autre jeu. 
%
Cette fois Barkisu voulait vérifier que Misha avec lu les derniers articles qu'elle avait
encodés et lancer à la mer.
%

-- Ah oui, j'ai vu cet article! Trois mille ans ou trois cent mille ans, quelle différence?

-- Héhé. En passant, as-tu réfléchis à votre nom: \textit{homo piscis}, \textit{homo aquaticus}? As-tu une préférence?

-- Oui, j'ai réfléchis. Je n'aime pas du tout l'idée d'un nom d'espèce qui fasse référence à mes branchies. Ce n'est pas là ce qui me distingue à mon avis. 
   Si homo sapiens veut dire \textit{humain sage} (ce qui est un assez gros mensonge), j'aimerais
   proposer \textit{homo scribis}, l'humain qui écrit. J'aime bien. Et en plus c'est la vérité!

C'était en effet la vérité. Depuis la génération 185, les specimen savaient lire et d'écrire de façon innée. C'était
la contribution de Djamila à ce projet. Une contribution immense. Encodage/décodage en cherchant la nourriture au fond de l'océan, etc.

-- Faudrait demander à ton frère ce qu'il en pense, mais j'aime assez.

Le terminal clignota pendant de longues minutes avant que Masha ne réponde. 

-- Maman. Tu sais que Yaya... tu sais qu'il ne me parle plus vraiment.

-- Oui je sais, je sais mon c\oe ur. Bon, on y va pour homo scribis alors. As-tu envie de travailler aujourd'hui?

Misha s'était avéré assez capable pour la génétique et le codage, mais ce n'était pas son passe temps préféré.
Le plus clair de son temps était plutôt dévolu à lire goulument et à écrire.

Barkisu se sentait bien seule. Vieille et seule.
Djamila, sa collaboratrice, sa compagne, l'amour de sa vie, n'était plus.
Seule dans cette gigantesque plate-forme flottante, jadis une ville bourdonnante.

Elle était encore blessée, elle qui avait tellement cru au projet \nomProjet{},
qui avait pour ainsi dire grandit dans ce projet. C'était sa vie.

Elle imaginait le monde autour d'elle descendu dans la guerre.

Elle imaginait, mais me préférait pas savoir, préférait continuer le travail.
Le travail de sa vie. 

De désespoir et de solitude, elle laissa un spécimen de la 300ième génération atteindre la maturité.
Elle 

Jusqu'au jour où elle a 

Il y a avait Yaya, le fils qu'elle avait perdu.
Il ne restait que Alpha, l'enfant qui lui restait.

\sautSection{}

Hermaprhodite

\sautSection{}

-- Misha, je t'aime!

Sous la mer, un grand filet s'ouvrit. Des millions d'oeufs fécondés coulèrent
doucement vers les profondeurs de l'océan.


