\chapter{Transition}

Barkisu alluma le terminal donnant sous la mer, un terminal immergé à près
vingt mètres de profondeur. 
%
Comme à chaque jour elle espérait que Misha vienne lui écrire quelque chose.  
%
Ou peut-être Yaya, même s'il n'avait pas écrit depuis des mois.  
% 
Encore aujourd'hui, penser à Yaya la faisait souffrir.  
%
Elle avait vraiment
peur pour lui. 
%
Plus encore, elle avait peur parce qu'il était devenu violent. 
%
Un de ces jours, il va s'en prendre à Misha, c'est évident.

-- Bonjour mère.

-- Misha, je suis tellement contente! Es-tu bien? Trouves-tu à manger?

Selon le protocole de recherche qu'elle avait elle-même élaboré avec Djamila,
elle n'aurait jamais dû laisser un spécimen atteindre la maturité. 
%
Encore moins deux!
%

-- C'est quelle génération aujourd'hui?

C'était un petit jeu qu'elle avait avec Misha, qui posait toujours cette
question même si son bébé connaissait tout à fait la réponse.
%
Peut-être que Misha voulait ainsi véfifier si Barkisu était encore en
possession de ses moyens.
%
C'est vrai qu'elle avait vraiment vieillie ces dernières années. 
%
Elle se sentait faible.
% 
Elle pensait beaucoup à Djamila, sa collaboratrice, l'amour de sa vie.
%
Elle rageait toujours que \nomProjet{} ce soit écrouler aussi rapidement, aussi
spéctaculairement.
%
Elle préférait ne pas penser au continent, mais elle l'imaginait consumé par la
guerre.
%
Un jour des bâteaux approcheraient la plate-forme à la recherche de quelque
chose de précieux: de la nourriture, des briques encores intactes, n'importe
quoi.
%
Un jour il faudra que son travail prenne fin. 
%
Un jour il s'agira de la dernière génération cultivée. 
%
Il faudra bien que la nature reprenne son cours
%


Barkisu se sentait bien seule. Vieille et seule.
Djamila, sa collaboratrice, sa compagne, l'amour de sa vie, n'était plus.
Seule dans cette gigantesque plate-forme flottante, jadis une ville bourdonnante.

Elle était encore blessée, elle qui avait tellement cru au projet \nomProjet{},
qui avait pour ainsi dire grandit dans ce projet. C'était sa vie.

Elle imaginait le monde autour d'elle descendu dans la guerre.

Elle imaginait, mais me préférait pas savoir, préférait continuer le travail.
Le travail de sa vie. 

De désespoir et de solitude, elle laissa un spécimen de la 300ième génération atteindre la maturité.
Elle 

Jusqu'au jour où elle a 

Il y a avait Yaya, le fils qu'elle avait perdu.
Il ne restait que Alpha, l'enfant qui lui restait.

\sautSection{}

-- Misha, je t'aime!

Sous la mer, un grand filet s'ouvrit. Des millions d'oeufs fécondés coulèrent
doucement vers les profondeurs de l'océan.


